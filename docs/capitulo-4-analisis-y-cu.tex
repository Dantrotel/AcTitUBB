\section{Introducción}

Este capítulo presenta el análisis funcional del sistema AcTitUBB mediante el enfoque de casos de uso, siguiendo la notación estándar del Lenguaje Unificado de Modelado (UML) \cite{uml_spec2017}. El análisis incluye la identificación de actores del sistema, la especificación de historias de usuario desde perspectivas ágiles, y la descripción detallada de los casos de uso críticos para el negocio.

El capítulo se estructura en las siguientes secciones:

\begin{itemize}
    \item \textbf{Historias de usuario:} Capturan las necesidades específicas desde la perspectiva de cada tipo de usuario (estudiante, profesor, administrador), siguiendo el formato ágil que facilita la comprensión de requisitos funcionales y su posterior validación.
    
    \item \textbf{Identificación de actores:} Define los roles que interactúan con el sistema, detallando sus responsabilidades organizacionales, nivel de conocimientos técnicos requeridos y privilegios de acceso según su función institucional.
    
    \item \textbf{Diagramas de casos de uso:} Modelan visualmente las interacciones principales entre actores y sistema, organizados por módulos funcionales (gestión de propuestas, autenticación, gestión de archivos).
    
    \item \textbf{Especificaciones detalladas:} Describen el flujo normal y alternativo de los casos de uso más críticos para el negocio, estableciendo precondiciones, postcondiciones y criterios de aceptación para cada funcionalidad.
\end{itemize}

La metodología utilizada combina el enfoque estructurado de UML para casos de uso \cite{uml_spec2017} con historias de usuario ágiles, proporcionando así una especificación completa que sirve tanto para el desarrollo técnico como para la validación con usuarios finales.


\section{Historias de Usuario}

Las historias de usuario describen las funcionalidades del sistema desde la perspectiva del usuario final, siguiendo el formato estándar: \textit{``Como [rol], quiero [acción] para [beneficio]''}. Cada historia incluye criterios de aceptación específicos que permiten su validación mediante pruebas funcionales. Las historias se organizan según los tres roles principales del sistema: Estudiante, Profesor y Administrador.

\subsection{Historias de Usuario - Estudiante}

El rol de Estudiante corresponde a los usuarios que crean y gestionan propuestas de título. Las siguientes historias capturan sus necesidades principales:

\textbf{HU-01:} Como estudiante de la UBB, quiero registrarme en el sistema con mi correo institucional para poder crear y gestionar mis propuestas de título.

\textbf{Criterios de aceptación:}
\begin{itemize}
    \item El sistema valida dominio \texttt{@alumnos.ubiobio.cl}.
    \item Envía confirmación por email.
    \item Asigna rol de estudiante automáticamente.
\end{itemize}

\textbf{HU-02:} Como estudiante autenticado, quiero crear una nueva propuesta de título para someter mi proyecto a revisión institucional.

\textbf{Criterios de aceptación:}
\begin{itemize}
    \item Formulario completo con título, descripción, fecha de envío.
    \item Carga opcional de archivos PDF/DOCX.
    \item Estado inicial ``Pendiente de revisión''.
\end{itemize}

\textbf{HU-03:} Como estudiante, quiero consultar el estado actual de mis propuestas para conocer el progreso del proceso de revisión.

\textbf{Criterios de aceptación:}
\begin{itemize}
    \item Vista de todas mis propuestas con estado actual.
    \item Fecha de última actualización.
    \item Comentarios del profesor si existen.
\end{itemize}

\textbf{HU-04:} Como estudiante, quiero recibir notificaciones automáticas para estar informado sobre cambios en mis propuestas.

\textbf{Criterios de aceptación:}
\begin{itemize}
    \item Emails automáticos cuando se asigna profesor.
    \item Notificación al cambiar estado o agregar comentarios.
\end{itemize}

\textbf{HU-05:} Como estudiante, quiero descargar archivos de revisión para acceder a la retroalimentación detallada de mi profesor.

\textbf{Criterios de aceptación:}
\begin{itemize}
    \item Enlaces de descarga disponibles cuando el profesor adjunta archivos de revisión.
\end{itemize}

\subsection*{Profesor}

\textbf{HU-06:} Como profesor de la UBB, quiero registrarme con mi correo institucional para acceder a las funciones de revisión.

\textbf{Criterios de aceptación:}
\begin{itemize}
    \item Validación de dominio \texttt{@ubiobio.cl}.
    \item Confirmación por email.
    \item Asignación automática de rol profesor.
\end{itemize}

\textbf{HU-07:} Como profesor revisor, quiero visualizar las propuestas asignadas a mí para gestionar mi carga de trabajo de revisiones.

\textbf{Criterios de aceptación:}
\begin{itemize}
    \item Lista filtrada solo con mis asignaciones.
    \item Información completa de cada propuesta.
    \item Archivos descargables.
\end{itemize}

\textbf{HU-08:} Como profesor revisor, quiero agregar comentarios detallados a las propuestas para proporcionar retroalimentación constructiva al estudiante.

\textbf{Criterios de aceptación:}
\begin{itemize}
    \item Editor de texto.
    \item Posibilidad de adjuntar archivos.
    \item Guardado automático.
    \item Notificación al estudiante.
\end{itemize}

\textbf{HU-09:} Como profesor revisor, quiero aprobar o rechazar propuestas para completar el proceso de evaluación.

\textbf{Criterios de aceptación:}
\begin{itemize}
    \item Solo disponible después de agregar comentarios.
    \item Cambio de estado irreversible.
    \item Notificaciones automáticas.
\end{itemize}

\textbf{HU-10:} Como profesor, quiero acceder al historial de mis revisiones para mantener seguimiento de mi actividad académica.

\textbf{Criterios de aceptación:}
\begin{itemize}
    \item Lista histórica de todas las propuestas revisadas.
    \item Fechas y estados finales.
\end{itemize}

\subsection*{Administrador}

\textbf{HU-11:} Como administrador del sistema, quiero gestionar usuarios registrados para mantener la integridad del sistema.

\textbf{Criterios de aceptación:}
\begin{itemize}
    \item CRUD completo de usuarios.
    \item Activación/desactivación de cuentas.
    \item Reseteo de contraseñas.
\end{itemize}

\textbf{HU-12:} Como administrador, quiero asignar profesores a propuestas para distribuir equitativamente la carga de revisiones.

\textbf{Criterios de aceptación:}
\begin{itemize}
    \item Lista de propuestas pendientes.
    \item Selección de profesores disponibles.
    \item Registro de asignación con timestamp.
\end{itemize}

\textbf{HU-13:} Como administrador, quiero visualizar estadísticas del sistema para monitorear el rendimiento y uso de la plataforma.

\textbf{Criterios de aceptación:}
\begin{itemize}
    \item Dashboard con métricas de propuestas por estado.
    \item Usuarios activos.
    \item Tiempo promedio de revisión.
\end{itemize}

\textbf{HU-14:} Como administrador, quiero configurar parámetros del sistema para adaptarlo a las necesidades institucionales.

\textbf{Criterios de aceptación:}
\begin{itemize}
    \item Configuración de tamaños de archivo.
    \item Formatos permitidos.
    \item Plantillas de notificaciones.
\end{itemize}

\section{Actores de casos de uso}

Los actores que interactúan con el sistema se detallan en la Tabla 13.

\begin{table}[H]
\centering
\renewcommand{\arraystretch}{1.5}
\begin{tabular}{ | m{2.7cm} | m{3.0cm} | m{3.0cm} | m{3.0cm} | m{3.0cm} | }
\hline
\textbf{Actor} & \textbf{Cargo(s)} & \textbf{Funciones en la organización} & \textbf{Nivel de conocimientos técnicos requeridos} & \textbf{Nivel de privilegio en el sistema} \\ \hline
\textbf{Estudiante} & Estudiante de pregrado UBB & Desarrollar y defender proyecto de título para obtener grado académico & Básico – Uso de navegador web, carga de archivos, formularios & Restringido – Solo acceso a sus propias propuestas \\ \hline
\textbf{Profesor} & Académico UBB, Profesor guía, Profesor revisor & Revisar, evaluar y aprobar propuestas de título de estudiantes asignados & Intermedio – Uso de plataformas web, gestión de archivos digitales & Medio – Acceso a propuestas asignadas, funciones de revisión \\ \hline
\textbf{Administrador} & Jefe de carrera, Coordinador académico & Gestionar proceso de títulos, asignar profesores, supervisar flujo académico & Avanzado – Administración de sistemas, configuración de parámetros & Alto – Acceso completo al sistema, gestión de usuarios y configuraciones \\ \hline
\textbf{Sistema de Email} & Servidor SMTP institucional & Envío automático de notificaciones y confirmaciones a usuarios & No aplica – Sistema automatizado & No aplica – Actor secundario \\ \hline
\end{tabular}
\caption{Actores del Sistema AcTitUBB}
\end{table}

\section{Diagramas y Especificación de casos de uso}

Incluya más de 1 diagrama para que quede claro el modelo.
Diagrama(s) de CU
\\
\textbf{Intercalar imagen con sus tablas de descripción}
\begin{figure}[H]
    \centering
    \IfFileExists{figures/caso_de_uso_gestion_propuestas.png}{%
        \includegraphics[scale=0.5]{figures/caso_de_uso_gestion_propuestas.png}%
    }{%
        \fbox{\parbox{0.8\linewidth}{\centering \textbf{Figura pendiente:} caso_de_uso_gestion_propuestas.png}}%
    }
    \caption{Diagrama de Casos de Uso - Propuestas}
    \label{fig:cu_propuestas}
\end{figure}
\begin{figure}[H]
    \centering
    \IfFileExists{figures/usuarios.png}{%
        \includegraphics[scale=0.5]{figures/usuarios.png}%
    }{%
        \fbox{\parbox{0.8\linewidth}{\centering \textbf{Figura pendiente:} usuarios.png}}%
    }
    \caption{Diagrama de Casos de Uso - Usuarios}
    \label{fig:cu_usuarios}
\end{figure}
\begin{figure}[H]
    \centering
    \IfFileExists{figures/caso_de_uso_gestion_archivos.png}{%
        \includegraphics[scale=0.5]{figures/caso_de_uso_gestion_archivos.png}%
    }{%
        \fbox{\parbox{0.8\linewidth}{\centering \textbf{Figura pendiente:} caso_de_uso_gestion_archivos.png}}%
    }
    \caption{Diagrama de Casos de Uso - Gestión de Archivos}
    \label{fig:cu_archivos}
\end{figure}

Liste los CU que están en su (s) diagramas destacando cuales serán detallados.
Considerando funcionalidad RELEVANTE del negocio especifique con la tabla sólo los CU relacionados. Para los CU restantes sólo incluya una descripción y precondiciones.

\section{Especificaciones Detalladas de Casos de Uso}

\subsection{CU\_01: Registrar Usuario}

\textbf{Precondiciones:} El usuario debe tener un correo electrónico institucional válido (@alumnos.ubiobio.cl o @ubiobio.cl) y acceso a internet.

\begin{table}[H]
    \centering
    \begin{tabular}{|p{7cm}|p{7cm}|}
        \hline
        \textbf{Actor} & \textbf{Aplicación} \\
        \hline
        \textbf{1.1)} El usuario accede a la página de registro del sistema. & \textbf{1)} El sistema despliega formulario de registro con campos: RUT, nombre completo, email, contraseña. \\
        \hline
        \textbf{1.2)} El usuario completa todos los campos requeridos con su información personal y correo institucional. & \textbf{3)} El sistema valida formato de RUT, email institucional y fortaleza de contraseña. Si la validación es correcta se llama al proceso "CU\_03: Confirmar Cuenta". \\
        \hline
        \textbf{1.3)} El usuario confirma el registro y acepta términos de uso. & \textbf{5)} El sistema asigna rol automáticamente según dominio del email, crea cuenta en estado "no confirmada" y envía email de confirmación. \\
        \hline
    \end{tabular}
\end{table}

\textbf{Flujo de eventos alternativos}

\begin{table}[H]
    \centering
    \begin{tabular}{|p{7cm}|p{7cm}|}
        \hline
        \textbf{Actor} & \textbf{Aplicación} \\
        \hline
        \textbf{1.2a)} El usuario ingresa email no institucional o datos con formato inválido. & \textbf{3.2a)} El sistema muestra errores específicos y mantiene datos válidos ingresados. \\
        \hline
        \textbf{1.3a)} El usuario ya existe en el sistema. & \textbf{5.3a)} El sistema informa que el usuario existe y ofrece opción de recuperar contraseña. \\
        \hline
    \end{tabular}
\end{table}

\textbf{Post condiciones:} El usuario queda registrado en estado "no confirmada" y recibe email con enlace de activación.

\subsection{CU\_02: Autenticar Usuario}

\textbf{Precondiciones:} El usuario debe estar registrado en el sistema y su cuenta debe estar confirmada.

\begin{table}[H]
    \centering
    \begin{tabular}{|p{7cm}|p{7cm}|}
        \hline
        \textbf{Actor} & \textbf{Aplicación} \\
        \hline
        \textbf{2.1)} El usuario accede a la página de inicio de sesión e ingresa sus credenciales. & \textbf{1)} El sistema despliega formulario de login con campos email/RUT y contraseña. \\
        \hline
        \textbf{2.2)} El usuario completa credenciales y selecciona "Iniciar Sesión". & \textbf{3)} El sistema valida credenciales contra la base de datos y verifica estado de confirmación de cuenta. \\
        \hline
        \textbf{2.3)} El usuario es redirigido al dashboard correspondiente a su rol. & \textbf{5)} El sistema genera token JWT, establece sesión y redirige según rol (estudiante/profesor/administrador). \\
        \hline
    \end{tabular}
\end{table}

\textbf{Flujo de eventos alternativos}

\begin{table}[H]
    \centering
    \begin{tabular}{|p{7cm}|p{7cm}|}
        \hline
        \textbf{Actor} & \textbf{Aplicación} \\
        \hline
        \textbf{2.2a)} El usuario ingresa credenciales incorrectas. & \textbf{3.2a)} El sistema muestra error de credenciales inválidas sin especificar cuál campo es incorrecto. \\
        \hline
        \textbf{2.2b)} La cuenta no está confirmada. & \textbf{3.2b)} El sistema informa que debe confirmar cuenta y ofrece reenvío de email. \\
        \hline
    \end{tabular}
\end{table}

\textbf{Post condiciones:} El usuario queda autenticado con sesión activa y token JWT válido.

\subsection{CU\_03: Confirmar Cuenta}

\textbf{Precondiciones:} El usuario debe estar registrado y haber recibido email de confirmación con enlace válido.

\begin{table}[H]
    \centering
    \begin{tabular}{|p{7cm}|p{7cm}|}
        \hline
        \textbf{Actor} & \textbf{Aplicación} \\
        \hline
        \textbf{3.1)} El usuario hace clic en el enlace de confirmación recibido por email. & \textbf{1)} El sistema valida el token de confirmación y verifica que no haya expirado. \\
        \hline
        \textbf{3.2)} El usuario confirma la activación de su cuenta. & \textbf{3)} El sistema actualiza estado de cuenta a "confirmada" y habilita acceso completo. \\
        \hline
        \textbf{3.3)} El usuario es redirigido a la página de login. & \textbf{5)} El sistema muestra mensaje de confirmación exitosa y llama a "CU\_13: Enviar Notificaciones" de bienvenida. \\
        \hline
    \end{tabular}
\end{table}

\textbf{Flujo de eventos alternativos}

\begin{table}[H]
    \centering
    \begin{tabular}{|p{7cm}|p{7cm}|}
        \hline
        \textbf{Actor} & \textbf{Aplicación} \\
        \hline
        \textbf{3.1a)} El enlace de confirmación ha expirado o es inválido. & \textbf{1.1a)} El sistema muestra error y ofrece opción de reenviar email de confirmación. \\
        \hline
    \end{tabular}
\end{table}

\textbf{Post condiciones:} La cuenta queda activada y el usuario puede iniciar sesión normalmente.

\subsection{CU\_04: Crear Propuesta}

\textbf{Precondiciones:} El usuario debe estar registrado como estudiante, logueado y con cuenta confirmada.

\begin{table}[H]
    \centering
    \begin{tabular}{|p{7cm}|p{7cm}|}
        \hline
        \textbf{Actor} & \textbf{Aplicación} \\
        \hline
        \textbf{4.1)} El estudiante accede al módulo de propuestas y selecciona "Nueva Propuesta". & \textbf{1)} El sistema despliega formulario de creación con campos obligatorios: título, descripción, fecha de envío. \\
        \hline
        \textbf{4.2)} El estudiante completa información requerida y opcionalmente adjunta archivo PDF/DOCX. & \textbf{3)} El sistema valida campos obligatorios y formato de archivo. Si es correcto llama a "CU\_06: Gestionar Archivos". \\
        \hline
        \textbf{4.3)} El estudiante confirma la creación de la propuesta. & \textbf{5)} El sistema asigna ID único, guarda en estado "Pendiente de revisión" y llama a "CU\_13: Enviar Notificaciones". \\
        \hline
    \end{tabular}
\end{table}

\textbf{Flujo de eventos alternativos}

\begin{table}[H]
    \centering
    \begin{tabular}{|p{7cm}|p{7cm}|}
        \hline
        \textbf{Actor} & \textbf{Aplicación} \\
        \hline
        \textbf{4.2a)} El estudiante adjunta archivo en formato no válido o muy grande. & \textbf{3.2a)} El sistema rechaza archivo y muestra formatos/tamaños permitidos. \\
        \hline
    \end{tabular}
\end{table}

\textbf{Post condiciones:} La propuesta se crea en estado "Pendiente de revisión" y se notifica al administrador.

\subsection{CU\_05: Consultar Estado Propuesta}

\textbf{Precondiciones:} El usuario debe estar logueado y tener permisos para acceder a la propuesta según su rol.

\begin{table}[H]
    \centering
    \begin{tabular}{|p{7cm}|p{7cm}|}
        \hline
        \textbf{Actor} & \textbf{Aplicación} \\
        \hline
        \textbf{5.1)} El usuario accede al módulo de consultas y selecciona una propuesta. & \textbf{1)} El sistema filtra propuestas según rol: estudiante ve propias, profesor asignadas, admin todas. \\
        \hline
        \textbf{5.2)} El usuario visualiza detalles, estado actual y comentarios disponibles. & \textbf{3)} El sistema muestra información según permisos del rol y historial de cambios. \\
        \hline
        \textbf{5.3)} El usuario descarga archivos disponibles según sus permisos. & \textbf{5)} El sistema permite descarga de archivos autorizados y llama a "CU\_06: Gestionar Archivos". \\
        \hline
    \end{tabular}
\end{table}

\textbf{Flujo de eventos alternativos}

\begin{table}[H]
    \centering
    \begin{tabular}{|p{7cm}|p{7cm}|}
        \hline
        \textbf{Actor} & \textbf{Aplicación} \\
        \hline
        \textbf{5.1a)} El usuario no tiene propuestas disponibles. & \textbf{1.1a)} El sistema muestra mensaje informativo de ausencia de propuestas. \\
        \hline
    \end{tabular}
\end{table}

\textbf{Post condiciones:} El usuario obtiene información actualizada y la consulta se registra en el log del sistema.

\subsection{CU\_06: Gestionar Archivos}

\textbf{Precondiciones:} El usuario debe estar autenticado y tener permisos para la operación de archivo solicitada.

\begin{table}[H]
    \centering
    \begin{tabular}{|p{7cm}|p{7cm}|}
        \hline
        \textbf{Actor} & \textbf{Aplicación} \\
        \hline
        \textbf{6.1)} El usuario selecciona operación de archivo: cargar, descargar o eliminar. & \textbf{1)} El sistema verifica permisos del usuario para la operación solicitada. \\
        \hline
        \textbf{6.2)} El usuario ejecuta la operación con el archivo seleccionado. & \textbf{3)} El sistema valida formato, tamaño y contenido del archivo. Ejecuta operación solicitada. \\
        \hline
        \textbf{6.3)} El usuario confirma que la operación se completó correctamente. & \textbf{5)} El sistema actualiza referencias en base de datos y registra operación en log de auditoría. \\
        \hline
    \end{tabular}
\end{table}

\textbf{Flujo de eventos alternativos}

\begin{table}[H]
    \centering
    \begin{tabular}{|p{7cm}|p{7cm}|}
        \hline
        \textbf{Actor} & \textbf{Aplicación} \\
        \hline
        \textbf{6.2a)} El archivo no cumple con las validaciones establecidas. & \textbf{3.2a)} El sistema rechaza operación y muestra detalles específicos del error. \\
        \hline
    \end{tabular}
\end{table}

\textbf{Post condiciones:} La operación de archivo se completa y queda registrada en el historial del sistema.

\subsection{CU\_07: Asignar Profesor}

\textbf{Precondiciones:} El usuario debe ser administrador logueado y debe existir propuesta en estado "Pendiente de revisión".

\begin{table}[H]
    \centering
    \begin{tabular}{|p{7cm}|p{7cm}|}
        \hline
        \textbf{Actor} & \textbf{Aplicación} \\
        \hline
        \textbf{7.1)} El administrador accede a gestión de propuestas y selecciona una pendiente. & \textbf{1)} El sistema muestra lista de propuestas pendientes con información básica. \\
        \hline
        \textbf{7.2)} El administrador visualiza detalles y selecciona "Asignar Profesor". & \textbf{3)} El sistema muestra lista de profesores disponibles con carga actual de trabajo. \\
        \hline
        \textbf{7.3)} El administrador selecciona profesor(es) y confirma asignación. & \textbf{5)} El sistema registra asignación con timestamp y llama a "CU\_13: Enviar Notificaciones". \\
        \hline
    \end{tabular}
\end{table}

\textbf{Flujo de eventos alternativos}

\begin{table}[H]
    \centering
    \begin{tabular}{|p{7cm}|p{7cm}|}
        \hline
        \textbf{Actor} & \textbf{Aplicación} \\
        \hline
        \textbf{7.3a)} El profesor seleccionado tiene sobrecarga de trabajo. & \textbf{5.3a)} El sistema advierte sobre sobrecarga pero permite confirmar asignación. \\
        \hline
    \end{tabular}
\end{table}

\textbf{Post condiciones:} La asignación se registra y se notifica a profesor y estudiante automáticamente.

\subsection{CU\_08: Revisar Propuesta}

\textbf{Precondiciones:} El usuario debe ser profesor logueado con propuestas asignadas para revisión.

\begin{table}[H]
    \centering
    \begin{tabular}{|p{7cm}|p{7cm}|}
        \hline
        \textbf{Actor} & \textbf{Aplicación} \\
        \hline
        \textbf{8.1)} El profesor accede a propuestas asignadas y selecciona una para revisar. & \textbf{1)} El sistema muestra lista de propuestas asignadas con estado actual. \\
        \hline
        \textbf{8.2)} El profesor revisa contenido y redacta comentarios detallados. & \textbf{3)} El sistema proporciona acceso a archivos y editor para comentarios. \\
        \hline
        \textbf{8.3)} El profesor adjunta archivos de retroalimentación y guarda revisión. & \textbf{5)} El sistema valida archivos, llama a "CU\_06: Gestionar Archivos" y "CU\_13: Enviar Notificaciones". \\
        \hline
    \end{tabular}
\end{table}

\textbf{Flujo de eventos alternativos}

\begin{table}[H]
    \centering
    \begin{tabular}{|p{7cm}|p{7cm}|}
        \hline
        \textbf{Actor} & \textbf{Aplicación} \\
        \hline
        \textbf{8.3a)} Error al guardar revisión por problema técnico. & \textbf{5.3a)} El sistema mantiene borrador de comentarios y permite reintento. \\
        \hline
    \end{tabular}
\end{table}

\textbf{Post condiciones:} Los comentarios se almacenan y el estudiante recibe notificación de nueva revisión.

\subsection{CU\_09: Aprobar/Rechazar Propuesta}

\textbf{Precondiciones:} El usuario debe ser profesor asignado, estar logueado, y la propuesta debe tener revisiones previas.

\begin{table}[H]
    \centering
    \begin{tabular}{|p{7cm}|p{7cm}|}
        \hline
        \textbf{Actor} & \textbf{Aplicación} \\
        \hline
        \textbf{9.1)} El profesor accede a propuesta revisada y selecciona "Decisión Final". & \textbf{1)} El sistema verifica comentarios previos y muestra opciones: Aprobar/Rechazar. \\
        \hline
        \textbf{9.2)} El profesor selecciona decisión y confirma que es irreversible. & \textbf{3)} El sistema solicita confirmación adicional advirtiendo sobre irreversibilidad. \\
        \hline
        \textbf{9.3)} El profesor confirma definitivamente la decisión. & \textbf{5)} El sistema actualiza estado final, crea proyecto si aprobada, y llama a "CU\_13: Enviar Notificaciones". \\
        \hline
    \end{tabular}
\end{table}

\textbf{Flujo de eventos alternativos}

\begin{table}[H]
    \centering
    \begin{tabular}{|p{7cm}|p{7cm}|}
        \hline
        \textbf{Actor} & \textbf{Aplicación} \\
        \hline
        \textbf{9.1a)} La propuesta no tiene comentarios de revisión. & \textbf{1.1a)} El sistema impide decisión y redirige a "CU\_08: Revisar Propuesta". \\
        \hline
    \end{tabular}
\end{table}

\textbf{Post condiciones:} La propuesta tiene estado final, se bloquea para edición y todos los actores reciben notificaciones.

\subsection{CU\_10: Gestionar Usuarios}

\textbf{Precondiciones:} El usuario debe ser administrador autenticado con permisos de gestión de usuarios.

\begin{table}[H]
    \centering
    \begin{tabular}{|p{7cm}|p{7cm}|}
        \hline
        \textbf{Actor} & \textbf{Aplicación} \\
        \hline
        \textbf{10.1)} El administrador accede al módulo de gestión de usuarios. & \textbf{1)} El sistema muestra lista de usuarios con filtros por rol y estado. \\
        \hline
        \textbf{10.2)} El administrador selecciona operación: crear, editar, activar o desactivar usuario. & \textbf{3)} El sistema despliega formulario correspondiente o solicita confirmación para la acción. \\
        \hline
        \textbf{10.3)} El administrador completa datos requeridos y confirma operación. & \textbf{5)} El sistema ejecuta operación, actualiza base de datos y llama a "CU\_13: Enviar Notificaciones" si corresponde. \\
        \hline
    \end{tabular}
\end{table}

\textbf{Flujo de eventos alternativos}

\begin{table}[H]
    \centering
    \begin{tabular}{|p{7cm}|p{7cm}|}
        \hline
        \textbf{Actor} & \textbf{Aplicación} \\
        \hline
        \textbf{10.3a)} Los datos ingresados son inválidos o el usuario ya existe. & \textbf{5.3a)} El sistema muestra errores específicos y mantiene datos válidos. \\
        \hline
    \end{tabular}
\end{table}

\textbf{Post condiciones:} La operación de gestión se completa y se registra en el log de auditoría del sistema.

\subsection{CU\_11: Ver Estadísticas}

\textbf{Precondiciones:} El usuario debe ser administrador autenticado con acceso al dashboard administrativo.

\begin{table}[H]
    \centering
    \begin{tabular}{|p{7cm}|p{7cm}|}
        \hline
        \textbf{Actor} & \textbf{Aplicación} \\
        \hline
        \textbf{11.1)} El administrador accede al módulo de estadísticas y reportes. & \textbf{1)} El sistema genera dashboard con métricas actualizadas: propuestas por estado, usuarios activos, tiempo promedio de revisión. \\
        \hline
        \textbf{11.2)} El administrador selecciona filtros de fecha o tipo de reporte específico. & \textbf{3)} El sistema aplica filtros y regenera estadísticas según criterios seleccionados. \\
        \hline
        \textbf{11.3)} El administrador exporta reportes o consulta detalles específicos. & \textbf{5)} El sistema genera archivo exportable o muestra vista detallada de métricas solicitadas. \\
        \hline
    \end{tabular}
\end{table}

\textbf{Flujo de eventos alternativos}

\begin{table}[H]
    \centering
    \begin{tabular}{|p{7cm}|p{7cm}|}
        \hline
        \textbf{Actor} & \textbf{Aplicación} \\
        \hline
        \textbf{11.2a)} No existen datos para los filtros seleccionados. & \textbf{3.2a)} El sistema informa ausencia de datos y sugiere ajustar criterios de búsqueda. \\
        \hline
    \end{tabular}
\end{table}

\textbf{Post condiciones:} El administrador obtiene información estadística actualizada y la consulta se registra.

\subsection{CU\_12: Configurar Sistema}

\textbf{Precondiciones:} El usuario debe ser administrador con permisos de configuración del sistema.

\begin{table}[H]
    \centering
    \begin{tabular}{|p{7cm}|p{7cm}|}
        \hline
        \textbf{Actor} & \textbf{Aplicación} \\
        \hline
        \textbf{12.1)} El administrador accede al módulo de configuración del sistema. & \textbf{1)} El sistema muestra panel de configuraciones: archivos permitidos, tamaños máximos, plantillas de email. \\
        \hline
        \textbf{12.2)} El administrador modifica parámetros según necesidades institucionales. & \textbf{3)} El sistema valida nuevos valores y muestra vista previa de cambios. \\
        \hline
        \textbf{12.3)} El administrador confirma aplicación de nueva configuración. & \textbf{5)} El sistema aplica cambios, actualiza configuración global y registra modificaciones en log de auditoría. \\
        \hline
    \end{tabular}
\end{table}

\textbf{Flujo de eventos alternativos}

\begin{table}[H]
    \centering
    \begin{tabular}{|p{7cm}|p{7cm}|}
        \hline
        \textbf{Actor} & \textbf{Aplicación} \\
        \hline
        \textbf{12.2a)} Los valores de configuración son inválidos o incompatibles. & \textbf{3.2a)} El sistema rechaza cambios y muestra restricciones específicas. \\
        \hline
    \end{tabular}
\end{table}

\textbf{Post condiciones:} La nueva configuración queda activa para todo el sistema y se registra el cambio.

\subsection{CU\_13: Enviar Notificaciones}

\textbf{Precondiciones:} Debe existir una acción en el sistema que requiera notificación y configuración SMTP válida.

\begin{table}[H]
    \centering
    \begin{tabular}{|p{7cm}|p{7cm}|}
        \hline
        \textbf{Actor} & \textbf{Aplicación} \\
        \hline
        \textbf{13.1)} El sistema detecta evento que requiere notificación (asignación, cambio de estado, nueva revisión). & \textbf{1)} El sistema identifica usuarios a notificar según tipo de evento y roles involucrados. \\
        \hline
        \textbf{13.2)} El sistema genera contenido personalizado del mensaje según plantilla correspondiente. & \textbf{3)} El sistema aplica plantilla específica e incluye datos relevantes del evento. \\
        \hline
        \textbf{13.3)} El sistema envía notificación por email y registra en sistema interno. & \textbf{5)} El sistema utiliza servidor SMTP configurado y registra éxito/fallo del envío en log. \\
        \hline
    \end{tabular}
\end{table}

\textbf{Flujo de eventos alternativos}

\begin{table}[H]
    \centering
    \begin{tabular}{|p{7cm}|p{7cm}|}
        \hline
        \textbf{Actor} & \textbf{Aplicación} \\
        \hline
        \textbf{13.3a)} Falla el envío de email por problema de conectividad o configuración SMTP. & \textbf{5.3a)} El sistema registra fallo, programa reintento automático y mantiene notificación en cola. \\
        \hline
    \end{tabular}
\end{table}

\textbf{Post condiciones:} Las notificaciones se envían exitosamente o quedan programadas para reintento, y se registra en log del sistema.