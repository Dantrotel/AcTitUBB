%!TEX root = ../informe.tex
\chapter{Especificación de Requerimientos del Producto de Software}\label{cap:requerimientos}
Este capítulo define formalmente los requerimientos del sistema AcTitUBB. Se delimitan el alcance y las fronteras del proyecto, se establecen las restricciones técnicas y se identifican los usuarios y sus objetivos. A continuación se listan los requerimientos funcionales (RF) y no funcionales (RNF), y se describen las interfaces de entrada y salida. La especificación deriva del análisis de objetivos institucionales para el proceso de titulación y de la observación de flujos disponibles en el repositorio actual del sistema (frontend Angular y backend Express + MySQL).

\section{Límites del Proyecto}\label{sec:limites}
El alcance de AcTitUBB cubre el ciclo de titulación desde la postulación de la propuesta hasta la coordinación de reuniones y la gestión de documentos finales. Se incluyen: creación y revisión de propuestas; conversión a proyectos aprobados; asignación de profesores en roles académicos; planificación (cronogramas e hitos); coordinación de reuniones mediante matching de disponibilidades; gestión de documentos asociados (propuesta, avances, entregable final) y notificaciones (internas y vía correo).\par
\textbf{No se incluyen} en el alcance actual:
\begin{itemize}
  \item Evaluación semántica o automática del contenido de documentos (plagio, calidad lingüística).
  \item Integración con LMS externo (Moodle, Canvas) ni repositorios institucionales (Dspace) para archivado permanente.
  \item Videoconferencia en tiempo real (se asume coordinación externa a la plataforma). 
  \item Analítica avanzada (dashboards predictivos de riesgo de atraso). 
  \item Internacionalización (i18n) más allá de español. 
\end{itemize}
\textbf{Fronteras}: el sistema interactúa únicamente con servicios internos (API REST) y el servidor de correo configurado vía \texttt{nodemailer}. La persistencia se centraliza en una instancia MySQL definida y gestionada por la institución.

\section{Restricciones Técnicas}\label{sec:restricciones}
\begin{itemize}
  \item \textbf{Plataforma Web}: Arquitectura cliente–servidor (Angular 20 + Express 5 + MySQL). 
  \item \textbf{Base de Datos}: Modelo relacional único (script consolidado \texttt{database.sql}). No se admite, por requerimientos institucionales, el uso de NoSQL.
  \item \textbf{Lenguajes y Frameworks}: TypeScript para frontend y partes del backend (si se tipifica), JavaScript/Node.js para lógica de servidor.
  \item \textbf{Autenticación}: Obligatoria vía JWT (Bearer) para rutas protegidas; sesión sin estado en backend.
  \item \textbf{Seguridad}: CORS restringido, rate limiting configurable (más estricto en producción). Uso de HTTPS exigido en despliegue.
  \item \textbf{Servidor de Correo}: Configuración SMTP debe ser provista externamente (credenciales válidas); sin correo no se envían notificaciones externas.
  \item \textbf{Compatibilidad Navegadores}: Últimas versiones estables de Chrome, Firefox, Edge. No se garantiza soporte para IE ni navegadores móviles antiguos.
  \item \textbf{Almacenamiento Archivos}: Sistema de archivos local (\texttt{uploads/}); no se incluye CDN ni almacenamiento distribuido.
  \item \textbf{Escalado}: Requerimientos de escalado horizontal limitados (carga institucional moderada); no se contempla balanceador dedicado en primera versión.
  \item \textbf{Tiempo de Entrega}: Cronograma delimitado por periodo académico, priorizando un MVP funcional sobre optimización exhaustiva.
\end{itemize}

\section{Identificación de Usuarios}\label{sec:usuarios}
\begin{description}
  \item[Estudiante] Actor principal que registra propuestas, consulta estado del proyecto, gestiona cronograma personal, carga documentos y solicita reuniones con profesores.
  \item[Profesor] Revisa propuestas en las que participa, atiende solicitudes de reunión, define disponibilidades, supervisa hitos y emite retroalimentación; puede ocupar múltiples roles (guía, co-guía, informante) según asignación.
  \item[Administrador] Configura parámetros globales, valida asignaciones de roles, supervisa la consistencia del modelo de datos y gestiona incidencias. Puede forzar ciertos cambios de estado.
\end{description}
\textbf{Objetivos por usuario}:
\begin{itemize}
  \item Estudiante: completar entrega final de titulación dentro de plazos.
  \item Profesor: guiar y evaluar progreso del estudiante, gestionar agenda académica.
  \item Administrador: asegurar cumplimiento de reglamentos, integridad y trazabilidad.
\end{itemize}

\section{Requerimientos Funcionales}\label{sec:req-funcionales}
Los requerimientos funcionales se codifican como RF-XX. Agrupados por módulos:
\subsection*{Autenticación y Seguridad}
\begin{itemize}
  \item RF-01: El sistema debe permitir inicio de sesión con credenciales institucionales y emitir un JWT válido. 
  \item RF-02: El sistema debe invalidar accesos a rutas protegidas sin token o con token expirado (respuesta 401). 
  \item RF-03: El sistema debe aplicar controles de rol (403) cuando un usuario accede a recursos no autorizados.
\end{itemize}
\subsection*{Gestión de Propuestas}
\begin{itemize}
  \item RF-04: El estudiante debe poder registrar una propuesta con título, resumen y objetivos. 
  \item RF-05: El estudiante debe poder editar una propuesta mientras esté en estado \texttt{Borrador}. 
  \item RF-06: El administrador debe poder aprobar o rechazar una propuesta con observaciones.
  \item RF-07: El sistema debe permitir listar propuestas filtradas por estado y autor.
\end{itemize}
\subsection*{Gestión de Proyectos}
\begin{itemize}
  \item RF-08: Al aprobar una propuesta, el sistema debe crear el proyecto asociado con estado inicial \texttt{En Curso}. 
  \item RF-09: El estudiante debe visualizar información consolidada del proyecto (roles asignados, hitos, documentos). 
  \item RF-10: El administrador debe cambiar el estado del proyecto a \texttt{Finalizado} o \texttt{Cancelado} según reglas.
\end{itemize}
\subsection*{Roles y Asignaciones}
\begin{itemize}
  \item RF-11: El sistema debe permitir asignar un profesor a un proyecto con rol específico (guía, co-guía, informante). 
  \item RF-12: El sistema debe impedir duplicar una asignación activa con el mismo \{proyecto, rol\}. 
  \item RF-13: El administrador debe poder remover o reemplazar asignaciones inválidas.
\end{itemize}
\subsection*{Cronograma e Hitos}
\begin{itemize}
  \item RF-14: El sistema debe permitir definir un cronograma asociado al proyecto. 
  \item RF-15: El estudiante debe registrar hitos con fecha planificada y descripción. 
  \item RF-16: El profesor debe actualizar el estado de cumplimiento del hito (pendiente, en revisión, aprobado). 
  \item RF-17: El sistema debe listar hitos por proyecto con filtros por estado.
\end{itemize}
\subsection*{Calendario y Reuniones (Matching)}
\begin{itemize}
  \item RF-18: El usuario debe registrar disponibilidades de horario (fecha, hora inicio y fin). 
  \item RF-19: El sistema debe permitir crear solicitud de reunión indicando al profesor y el proyecto. 
  \item RF-20: El sistema debe procesar búsqueda de reunión (matching) cruzando disponibilidades y sugerir franja viable. 
  \item RF-21: El profesor debe aceptar o rechazar la solicitud propuesta. 
  \item RF-22: El sistema debe registrar historial de reuniones (creación, reprogramación, cancelación).
\end{itemize}
\subsection*{Gestión de Documentos}
\begin{itemize}
  \item RF-23: El estudiante debe subir documentos (propuesta final, avances, versión final) vinculados al proyecto. 
  \item RF-24: El sistema debe almacenar metadatos del documento (tipo, versión, fecha). 
  \item RF-25: El profesor debe revisar y cambiar estado (aprobado, requiere corrección). 
  \item RF-26: El sistema debe permitir descargar documentos aprobados.
\end{itemize}
\subsection*{Notificaciones y Auditoría}
\begin{itemize}
  \item RF-27: El sistema debe generar notificación interna ante eventos críticos (aprobación propuesta, reunión aceptada, documento revisado). 
  \item RF-28: El sistema debe enviar correo al destinatario según criticidad configurada. 
  \item RF-29: El usuario debe marcar notificaciones como leídas. 
  \item RF-30: El sistema debe mantener registros de auditoría para acciones críticas (asignación rol, confirmación reunión).
\end{itemize}
\subsection*{Búsqueda y Filtrado}
\begin{itemize}
  \item RF-31: El usuario debe filtrar listados por estado (propuestas, proyectos, hitos, solicitudes). 
  \item RF-32: El sistema debe soportar paginación en listados extensos.
\end{itemize}

\section{Requerimientos No Funcionales}\label{sec:req-nofuncionales}
Codificados como RNF-XX.
\subsection*{Rendimiento}
\begin{itemize}
  \item RNF-01: Las respuestas de endpoints principales (login, listar propuestas, matching) deben completarse en < 500 ms en entorno estándar de laboratorio.
  \item RNF-02: Operaciones de subida de documentos deben concluir sin bloqueo del hilo principal (manejo asíncrono). 
\end{itemize}
\subsection*{Seguridad}
\begin{itemize}
  \item RNF-03: Todo acceso a recursos protegidos debe requerir autorización JWT válida. 
  \item RNF-04: El sistema debe limitar intentos de autenticación (rate limiting) para mitigar fuerza bruta. 
  \item RNF-05: Los documentos deben almacenarse fuera del directorio público y referenciarse por ID interno.
\end{itemize}
\subsection*{Disponibilidad y Confiabilidad}
\begin{itemize}
  \item RNF-06: El sistema debe tolerar reinicios del backend sin pérdida de información persistente. 
  \item RNF-07: Las operaciones críticas deben registrar auditoría para recuperación de contexto.
\end{itemize}
\subsection*{Escalabilidad}
\begin{itemize}
  \item RNF-08: La arquitectura debe permitir desplegar múltiples instancias del backend detrás de un balanceador sin requerir sesión de estado.
\end{itemize}
\subsection*{Mantenibilidad}
\begin{itemize}
  \item RNF-09: El código debe estar modularizado (servicios, controladores, modelos) para facilitar refactor. 
  \item RNF-10: Se debe mantener un único script SQL maestro para reproducir la base de datos.
\end{itemize}
\subsection*{Usabilidad}
\begin{itemize}
  \item RNF-11: La interfaz debe proveer mensajes claros de error y estados vacíos indicando acción sugerida. 
  \item RNF-12: Formularios deben validar campos requeridos antes de habilitar envío.
\end{itemize}
\subsection*{Portabilidad}
\begin{itemize}
  \item RNF-13: El backend debe ejecutarse en entornos Windows y Linux con mínima reconfiguración (ajuste de rutas y variables de entorno). 
\end{itemize}
\subsection*{Compatibilidad}
\begin{itemize}
  \item RNF-14: El frontend debe soportar las versiones vigentes de los principales navegadores (Chrome, Firefox, Edge) sin degradación funcional.
\end{itemize}

\section{Interfaces Externas de Entrada}\label{sec:interfaces-entrada}
\subsection*{Formularios Web}
\begin{itemize}
  \item Registro de propuesta (título, objetivos, resumen). 
  \item Registro de disponibilidad (fecha, hora inicio, hora fin). 
  \item Subida de documentos (selección archivo, tipo documento). 
  \item Creación de solicitud de reunión (proyecto, profesor, motivo). 
  \item Asignación de profesor (selección rol, profesor, proyecto). 
\end{itemize}
\subsection*{Endpoints REST}
Autenticación: \texttt{POST /api/v1/auth/login} (credenciales).\par
Propuestas: \texttt{POST /api/v1/propuestas}, \texttt{GET /api/v1/propuestas?estado=...}.\par
Proyectos: \texttt{POST /api/v1/proyectos}, \texttt{GET /api/v1/projects/:id}.\par
Asignaciones: \texttt{POST /api/v1/asignaciones-profesores}.\par
Calendario Matching: \texttt{POST /api/v1/calendario-matching/disponibilidades}, \texttt{POST /api/v1/calendario-matching/solicitudes}, \texttt{POST /api/v1/calendario-matching/buscar-reunion}.\par
Documentos: \texttt{POST /api/v1/documentos}.\par
Notificaciones: \texttt{GET /api/v1/notificaciones}.\par
\subsection*{Cabeceras y Autenticación}
La mayoría de las rutas requieren \texttt{Authorization: Bearer <JWT>}. Se usan cabeceras estándar (\texttt{Content-Type: application/json}), excepto subida de archivos (\texttt{multipart/form-data}).

\section{Interfaces Externas de Salida}\label{sec:interfaces-salida}
\subsection*{Respuestas JSON}
Patrón base:
\begin{verbatim}
{
  "data": { ... },
  "message": "Descripción opcional",
  "error": null
}
\end{verbatim}
Errores:
\begin{verbatim}
{
  "data": null,
  "message": "Detalle del error",
  "error": {
     "code": 400,
     "details": ["campo X requerido"]
  }
}
\end{verbatim}
\subsection*{Descarga de Documentos}
El endpoint de documentos provee contenido binario con cabecera \texttt{Content-Disposition: attachment; filename="..."}. Se filtra acceso según rol y estado (versión aprobada).
\subsection*{Notificaciones Internas}
Listas en frontend con campos: id, mensaje, tipo (info, warning, critical), estado (leída/no leída), fecha. Acciones de cierre emiten actualización vía respuesta JSON. 
\subsection*{Correos Electrónicos}
Formato de correo: asunto (evento), cuerpo HTML simple (resumen + enlace), remitente institucional. Errores SMTP no bloquean la operación principal (se registran en auditoría).
\subsection*{Indicadores en UI}
Snackbars y badges (número de notificaciones no leídas) complementan salida hacia usuario, mejorando la percepción de estado.

\bigskip
Esta especificación crea una base verificable para validar la implementación existente y futuras extensiones. Cada requerimiento y restricción está vinculado a un módulo concreto, permitiendo trazabilidad hacia casos de prueba y código fuente.
