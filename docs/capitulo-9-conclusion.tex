%!TEX root = ../informe.tex
\chapter{Conclusiones}\label{cap:conclusiones}
Este capítulo presenta la síntesis final del proyecto AcTitUBB y plantea líneas de trabajo futuro que permitirían ampliar su impacto institucional y técnico. Las conclusiones se derivan de la articulación entre requerimientos, diseño, desarrollo e integración operativa observada en los capítulos previos.

\section{Conclusión General}\label{sec:conclusion-general}
AcTitUBB entrega una plataforma integral para gestionar el ciclo de titulación, desde la postulación de propuestas hasta la coordinación de reuniones y manejo de entregables finales. A lo largo del desarrollo se consolidó una arquitectura modular (Angular 20 en frontend; Express 5 + MySQL en backend) que soporta las principales funciones académicas: validación y transición de propuestas a proyectos, asignación de roles docentes, definición y seguimiento de hitos, calendarización con matching de disponibilidades, versionado de documentos y notificaciones internas/externas.

Los objetivos planteados se cumplieron al: (i) unificar el modelo de datos en un script maestro que centraliza la persistencia y reduce ambigüedades; (ii) establecer flujos claros de autenticación y autorización con JWT y control de rol; (iii) materializar un subsistema de reuniones que cruza disponibilidad y genera trazabilidad mediante historial; (iv) asegurar trazabilidad en acciones críticas con auditoría y métricas cualitativas de rendimiento satisfactorias; y (v) ofrecer una interfaz coherente que separa funciones por rol (estudiante, profesor, administrador) con estados vacíos, validaciones y feedback oportuno.

Las incidencias enfrentadas —desalineación de rutas, referencias a columnas inexistentes, ajuste de validaciones de rol, unificación de scripts SQL y relajación del rate limiting en desarrollo— impulsaron mejoras que fortalecen la mantenibilidad. El código resultante evidencia separación de capas, control explícito de efectos laterales (subida de archivos, correo) y bases para escalamiento horizontal del backend.

Se reconoce como limitación la ausencia de pruebas automatizadas extensivas y mediciones cuantitativas formales (benchmark de concurrencia, estrés sobre endpoints de matching). Aun así, la robustez funcional y la alineación a requerimientos institucionales posicionan la solución como un cimiento sólido para la gestión académica del proceso de titulación.

En síntesis, el proyecto logra integrar componentes esenciales del flujo de titulación, reduce fricciones operativas y prepara una estructura que facilita extensiones futuras sin comprometer la coherencia del diseño existente.

\section{Trabajo Futuro (Sólo Informe Final)}\label{sec:trabajo-futuro}
Existen oportunidades de evolución orientadas a ampliar cobertura funcional, calidad técnica y valor estratégico:
\begin{enumerate}
  \item \textbf{Pruebas Automatizadas y Calidad}: Incorporar baterías de pruebas unitarias (servicios y controladores), pruebas de integración sobre endpoints críticos y pruebas end-to-end (Cypress / Playwright) para reducir riesgo de regresiones. Integrar análisis estático (ESLint estricto, SonarQube) y pipeline CI/CD.
  \item \textbf{Métricas y Observabilidad}: Añadir instrumentación (APM, logs estructurados, métricas Prometheus) para medir tiempo de respuesta, frecuencias de error y uso de funcionalidades (reuniones, hitos). Esto permitiría retroalimentar decisiones de mejora y capacidad.
  \item \textbf{Analítica Predictiva}: Desarrollar módulo de analítica que calcule riesgo de atraso por correlación entre cumplimiento de hitos, reprogramaciones de reuniones y entregas de documentos. Modelos simples de clasificación podrían priorizar intervenciones tempranas del profesor guía.
  \item \textbf{Gestión Documental Avanzada}: Integrar almacenamiento externo (S3, Azure Blob) con control de versiones más granular, firma digital de entregas y validaciones automáticas de formato (plantillas institucionales, verificación de metadatos).
  \item \textbf{Integración Institucional}: Conectar con sistemas de gestión académica (SIGA / matrícula) para sincronizar datos de usuario y estados de inscripción; APIs hacia repositorios de investigación para archivado final y DOI.
  \item \textbf{Extensión de Matching}: Incorporar algoritmos de optimización (interval scheduling, heurísticas multi-objetivo) para sugerir la mejor franja considerando preferencias de ambos actores y evitar saturación de horarios.
  \item \textbf{Escalabilidad y Despliegue}: Migrar a contenedores orquestados (Kubernetes), configurar autoescalado basado en métricas y reforzar la estrategia de configuración por entorno (secretos en vault seguro).
  \item \textbf{Internacionalización y Accesibilidad}: Implementar i18n para otros idiomas y auditoría de accesibilidad (WCAG 2.1 AA) con mejoras en contraste, navegación via teclado y etiquetas ARIA enriquecidas.
  \item \textbf{Gestión de Flujos Avanzados}: Añadir revisor externo (pare evaluador), rondas de retroalimentación formal con timestamps y comparative tracking de versiones de documento (diff visual).
  \item \textbf{Panel Analítico de Estado Institucional}: Dashboard administrativo con agregados de porcentaje de avance global, distribución de estados de proyectos y mapa de carga docente para balancear asignaciones futuras.
  \item \textbf{Automatización de Recordatorios}: Motor de eventos que envíe recordatorios antes de fechas importantes e hitos próximos a vencer, con lógica configurable por umbral.
  \item \textbf{Seguridad Avanzada}: Rotación automática de secretos JWT, detección de comportamientos anómalos (multiple login fallido), cifrado de documentos en reposo y análisis de vulnerabilidades periódicas.
\end{enumerate}
La prioridad inicial debería enfocarse en pruebas automatizadas y observabilidad para cimentar confianza operativa; posteriormente, analítica predictiva y paneles administrativos brindarían valor estratégico. Las mejoras propuestas aprovechan la modularidad actual y se integran con disrupción mínima en las capas existentes.

\bigskip
La proyección futura muestra un camino claro hacia una plataforma institucional madura, escalable y analíticamente informada, consolidando a AcTitUBB como herramienta clave en la gestión de titulación.
