% Archivo generado: estimacion_gestion.tex
% Contiene: Carta Gantt (línea base y desviaciones), Riesgos, Estimación CU (UCP)
% y Resumen de Esfuerzo. Copiar/pegar en el documento principal según convenga.
\documentclass[12pt]{article}
\usepackage[utf8]{inputenc}
\usepackage[T1]{fontenc}
\usepackage[spanish]{babel}
\usepackage{longtable}
\usepackage{array}
\usepackage{booktabs}
\usepackage{geometry}
\usepackage{caption}
\geometry{margin=2.5cm}
\begin{document}

\section{Carta Gantt con línea base y desviaciones}

% Tabla alternativa al diagrama Gantt que describe actividades, línea base y desviaciones
\begin{longtable}{|p{3.5cm}|p{2.8cm}|p{2.8cm}|p{5.0cm}|}
\hline
\textbf{Actividad / Hito} & \textbf{Línea base (fecha inicio - fin)} & \textbf{Fecha real (inicio - fin)} & \textbf{Explicación del cambio y efecto en la planificación global} \\
\hline
Configuración inicial del repositorio & 2025-08-10 - 2025-08-16 & 2025-08-10 - 2025-08-16 & Entrega dentro de la línea base. Ningún efecto en planificación. \\
\hline
Análisis de requisitos (Cap.3) & 2025-08-25 - 2025-09-02 & 2025-08-25 - 2025-09-01 & Entrega un día antes; solicitadas aclaraciones sobre actores y trazabilidad — añade 1 día de buffer para revisión. \\
\hline
Diseño del modelo de datos & 2025-09-05 - 2025-09-12 & 2025-09-05 - 2025-09-10 & Acelerado por decisiones de alcance; impacto positivo en fases posteriores (menor rework). \\
\hline
Autenticación y Registro & 2025-09-25 - 2025-10-02 & 2025-09-30 - 2025-10-05 & Retraso 3 días por cambios en validación de RUT y política de email institucional; replanificar tareas dependientes (creación de propuestas) +2 días. \\
\hline
Gestión de Propuestas & 2025-10-05 - 2025-10-18 & 2025-10-07 - 2025-10-20 & Desviación +2 días por integración de subida de archivos y validaciones; efecto: desplazar revisión y asignación 2 días. \\
\hline
Asignaciones y Revisión & 2025-10-20 - 2025-11-01 & 2025-10-22 - 2025-11-01 & Menor impacto; ajustes en índice único para evitar duplicados realizados durante la iteración. \\
\hline
Notificaciones Email y Colas & 2025-10-28 - 2025-11-05 & 2025-10-28 - 2025-11-05 & Se detectaron problemas SMTP en entorno dev; se usa MailHog para prueba. Impacto mínimo en funcionalidad, mayor esfuerzo en pruebas. \\
\hline
Creación automática de Proyecto & 2025-11-06 - 2025-11-10 & 2025-11-08 - 2025-11-11 & Requiere ajustes para rollback; desviación +1 día. \\
\hline
Gestión Administrativa y SuperAdmin & 2025-11-12 - 2025-11-18 & 2025-11-12 - 2025-11-18 & Se añadió HU/CU para SuperAdmin; no afecta la fecha pero agrega alcance de documentaci\'on. \\
\hline
Documentación y CAP (Cap.4) & 2025-11-18 - 2025-11-24 & 2025-11-18 - 2025-11-24 & Correcciones LaTeX y creación de contenedor CAP; se añadieron fallbacks para figuras, sin impacto en entregables. \\
\hline
Pruebas de Aceptación (planificación) & 2025-11-25 - 2025-11-30 & 2025-11-25 - 2025-11-26 & Se generaron tablas de pruebas; ejecución completa pendiente (staging). \\
\hline
\caption{Carta Gantt (tabla) — Línea base vs ejecución real y desviaciones}
\end{longtable}

\bigskip
\noindent\textbf{Observaciones:}
\begin{itemize}
    \item Las desviaciones detectadas son mayoritariamente de alcance (nuevos requisitos) o integraci\'on (SMTP, triggers). Se recomienda mantener un buffer del 10-15\% para la siguiente iteración.
    \item Los cambios que afectan a la integridad de datos (triggers y rollback) deben preceder a despliegues a staging.
\end{itemize}

\section{Riesgos de Alto Nivel (Amenazas), Impacto y Estrategia}

\begin{longtable}{|p{4.0cm}|p{2.0cm}|p{2.4cm}|p{6.2cm}|p{2.0cm}|}
\hline
\textbf{Riesgo} & \textbf{Impacto} & \textbf{Probabilidad} & \textbf{Estrategia / Acciones propuestas} & \textbf{Presentado?} \\
\hline
Fallo del servicio SMTP (env\'io de notificaciones) & Alto & Media & Usar servidor SMTP de pruebas (MailHog) en dev/staging; implementar \texttt{email\_queue} con reintentos y monitoreo; alertas en production. & S\'i (dev) \\
\hline
Ausencia de SuperAdmin y operaciones globales (gap de requisitos) & Alto & Media & Añadir rol \texttt{superadmin} y CU/HU; proteger operaciones críticas mediante confirmaciones y auditor\'ia; bloquear eliminación del \"\'ultimo\" SuperAdmin. & S\'i (documentado) \\
\hline
Falta de triggers/rollback para migraciones y acciones masivas & Alto & Media & Crear \texttt{migrations\_executions}, ejecutar migraciones en transacciones, pruebas de preview y rollback. & Identificado \\
\hline
Problemas de integridad al crear proyecto automáticamente & Medio-Alto & Baja-Media & Revisar constraints (unique proyecto\_id), añadir manejo de errores y logging; testear trigger en datos reales. & S\'i (trigger implementado; revisar) \\
\hline
Carga y reintentos de jobs (colas) sin control & Medio & Media & Implementar \texttt{job\_queue} con prioridad/reintentos y vistas de monitoreo; políticas de backoff. & Identificado \\
\hline
Dependencia en RUT y validaciones específicas (posible rechazo por usuarios) & Medio & Media & Documentar formato RUT, mensajes claros en UI, utilitarios de validación en backend; pruebas de aceptación. & S\'i \\
\hline
Problemas de compilaci\'on LaTeX por figuras faltantes & Bajo-Medio & Alta & Usar \texttt{\textbackslash IfFileExists} (ya aplicado) y generar placeholders autom\'aticos si se desea. & S\'i \\
\hline
Falta de cobertura de pruebas de aceptación completas en staging & Alto & Media & Preparar dataset de prueba y scripts autom\'aticos; ejecutar regresiones tras cambios. & Identificado \\
\hline
\caption{Riesgos, impacto, estrategia y si han ocurrido}
\end{longtable}

\section{Estimación CU (Use Case Points - UCP)}

\subsection{Resumen del m\'etodo}
Se utiliza el m\'etodo UCP (Unadjusted Use Case Points) con los pasos:
\begin{enumerate}
    \item Clasificar actores y asignar peso (Simple=1, Medio=2, Complejo=3)
    \item Clasificar casos de uso y asignar peso (Simple=5, Medio=10, Complejo=15)
    \item Calcular UAW (suma pesos actores) y UUCW (suma pesos casos de uso)
    \item UUCP = UAW + UUCW
    \item Calcular TCF (Technical Complexity Factor): TFactor = $\sum$ (multiplicador * relevancia)
    \quad TCF = 0.6 + (0.01 * TFactor)
    \item Calcular EF (Environmental Factor): EFactor = $\sum$ (multiplicador * relevancia)
    \quad EF = 1.4 + (-0.03 * EFactor)
    \item UCP = UUCP * TCF * EF
\end{enumerate}

\subsection{Clasificaci\'on de actores (ejemplo tomado del proyecto)}
\begin{tabular}{|p{5.0cm}|p{3.0cm}|p{6.0cm}|}
\hline
Actor & Tipo & Peso \\
\hline
Estudiante (interacci\'on por GUI web) & Complejo & 3 \\
Profesor (GUI web) & Complejo & 3 \\
Administrador (GUI web) & Complejo & 3 \\
SuperAdmin (GUI web, operaciones avanzadas) & Complejo & 3 \\
\hline
\end{tabular}

\noindent\textbf{UAW = 3 + 3 + 3 + 3 = 12}

\subsection{Clasificaci\'on de casos de uso (estimaci\'on ejemplo)}
\begin{tabular}{|p{6.0cm}|p{2.0cm}|p{4.0cm}|}
\hline
Caso de uso & Tipo & Peso \ \\
\hline
Registrar usuario & Simple & 5 \\
Login / Autenticaci\'on & Simple & 5 \\
Crear propuesta (subida archivos) & Medio & 10 \\
Asignar profesores & Medio & 10 \\
Revisar propuesta (aprobar/rechazar/correcciones) & Complejo & 15 \\
Subir avances & Medio & 10 \\
Descargar archivos protegidos & Simple & 5 \\
Gestor de usuarios (CRUD admin) & Medio & 10 \\
Configuraci\'on estructura acad\'emica (SuperAdmin) & Complejo & 15 \\
Migraciones con preview/rollback & Complejo & 15 \\
Logs y auditor\'ia (export) & Medio & 10 \\
Operaciones de emergencia (reintentos colas) & Medio & 10 \\
\hline
\end{tabular}

\noindent\textbf{UUCW = 3*5 + 5*10 + 3*15 = 15 + 50 + 45 = 110 (ejemplo)}

\noindent\textbf{UUCP = UAW + UUCW = 12 + 110 = 122}

\subsection{C\'alculo TCF (ejemplo con relevancias asignadas)}
\begin{tabular}{|p{6.5cm}|p{2.0cm}|p{2.8cm}|p{2.8cm}|}
\hline
Factor T\'ecnico & Multiplicador & Relevancia (0..5) & Resultado (mult * rel) \\
\hline
Distributed System & 2 & 2 & 4 \\
Performance objectives & 1 & 3 & 3 \\
End-user efficiency (on-line) & 1 & 3 & 3 \\
Complex internal processing & 1 & 4 & 4 \\
Reusability & 1 & 2 & 2 \\
Installation ease & 0.5 & 1 & 0.5 \\
Operational ease & 0.5 & 3 & 1.5 \\
Portability & 2 & 2 & 4 \\
Changeability & 1 & 2 & 2 \\
Concurrency & 1 & 3 & 3 \\
Security features & 1 & 2 & 2 \\
Direct access for third parties & 1 & 1 & 1 \\
Special user training facilities & 1 & 1 & 1 \\
\hline
\multicolumn{3}{|r|}{\textbf{TFactor =}} & 31.0 \\
\hline
\end{tabular}

\noindent\textbf{TCF = 0.6 + (0.01 * TFactor) = 0.6 + 0.31 = 0.91}

\subsection{C\'alculo EF (ejemplo con relevancias asignadas)}
\begin{tabular}{|p{7.5cm}|p{2.5cm}|p{3.0cm}|}
\hline
Factor Ambiental & Multiplicador & Relevancia (0..5) \\
\hline
Familiar with Objectory + RUP & 1.5 & 3 \\
Application experience & 0.5 & 3 \\
Object Oriented experience & 1 & 3 \\
Analyst capability & 0.5 & 3 \\
Motivation & 1 & 4 \\
Stable requirements & 2 & 2 \\
Part time workers & -1 & 1 \\
Difficult programming language & -1 & 1 \\
\hline
\multicolumn{2}{|r|}{\textbf{EFactor (suma ponderada) =}} & 16.5 \\
\hline
\end{tabular}

\noindent\textbf{EF = 1.4 + (-0.03 * EFactor) = 1.4 - 0.495 = 0.905}

\subsection{UCP y Esfuerzo estimado (ejemplo)}
\noindent UCP = UUCP * TCF * EF = 122 * 0.91 * 0.905 $\approx$ 100.6 \newline
\noindent Redondeando: UCP \approx 101 puntos.

\noindent Para convertir UCP a horas se usa un factor de conversion (por ejemplo 20 horas por UCP como referencia):
\begin{itemize}
    \item \textbf{Horas estimadas} = UCP * 20 = 101 * 20 = 2020 horas (ejemplo; ajustar seg\'un factibilidad del equipo).
\end{itemize}

\subsection{Nivel de Esfuerzo (Schneider \\& Winters) - Regla rápida}
\begin{itemize}
    \item Si (n\'umero de factores de entorno F1..F6 inferiores a 3) + (n\'umero de factores F7..F8 superiores a 3) \le 2  entonces LOE = 20
    \item Si es 3 o 4 entonces LOE = 28
    \item Si es mayor a 4 reconsiderar el proyecto (reducir riesgos o ajustar alcance)
\end{itemize}

\section{Resumen de Esfuerzo}

% Plantilla para rellenar horas reales por actividad/fase
\begin{longtable}{|p{8.5cm}|p{4.0cm}|}
\hline
\textbf{Actividad / Fase / Caso de Uso} & \textbf{N\'umero de horas} \\
\hline
Análisis de requisitos y documentación & 120 \\
Dise\~no del modelo de datos y ER & 80 \\
Implementación: Autenticación y Registro & 160 \\
Implementación: Gestión de Propuestas y Subida de archivos & 240 \\
Implementación: Asignaciones y Revisión & 180 \\
Implementación: Notificaciones y Colas & 120 \\
Implementación: Creación automática de Proyecto & 80 \\
Implementación: Gestión Administrativa y SuperAdmin & 140 \\
Pruebas unitarias e integración & 200 \\
Documentación (CAP, casos de uso, pruebas) & 120 \\
Control de versiones / tareas git & 60 \\
Reuniones y coordinación & 120 \\
\hline
\textbf{TOTAL} & \textbf{~1620} \\
\hline
\caption{Resumen ejemplo de horas por fase (ajustar con datos reales del equipo)}
\end{longtable}

\bigskip
\noindent\textbf{Cómo usar este documento}
\begin{itemize}
    \item Copiar este archivo al proyecto LaTeX principal y ajustar las fechas y valores num\'ericos con datos reales obtenidos del seguimiento (git commits, hojas de tiempo).
    \item Para la Carta Gantt real, puede usar paquetes como \texttt{pgfgantt} si prefiere un diagrama visual; aquí se incluye una tabla legible que documenta desviaciones y causas.
    \item Revisar y afinar las relevancias de factores TCF/EF con el equipo para obtener una estimaci\'on UCP lo m\'as realista posible.
\end{itemize}

\end{document}
